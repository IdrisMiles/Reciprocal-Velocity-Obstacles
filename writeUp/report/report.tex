\documentclass[a4paper,twocolumn]{article}
%\documentclass[a4paper]{article}
\usepackage{graphicx}
\usepackage{caption}
\usepackage{subcaption}
\usepackage{dblfloatfix}
\usepackage{fixltx2e}
\usepackage{url}
\usepackage{amssymb}
%\usepackage{natbib}

%\addbibresource[<options for bib resources>]{<mybibfile>.bib}


%Set up title
\author{Idris Miles\\
Bournemouth University SDAGE level I}
\title{Local Avoidance Techniques for Real-Time Crowd Simulation}
\date{\today}

%begin the document
\begin{document}

\maketitle

\section*{Abstract}
In this paper I demonstrate the use of Reciprocal Velocity Obstacle, RVO, as a viable local avoidance model for real time agent based crowd simulation. \\

\section{Introduction}
Crowd simulation is a growing in films and games, where complete CG worlds are brought to life with their own CG inhabitants. Of course it is not feasible to animate 300 pedestrians in the background of a 10 second shot, or thousands of fans in a sports arena. So the use of crowd simulation tools were bought to life.

\section{Local Avoidance in Crowd Simulation}
Local avoidance is the avoidance of agents within close proximity of each other, it is useful when there are multiple moving objects in the world that the global navigation system may ignore. Global navigation differs from local avoidance because it looks at finding a route from one location to another, while local avoidance just considers its surroundings and desired direction inn order to avoid collisions while steering towards its goal. Artificial Intelligence also differs from local avoidance although the two can be linked together, AI focuses on creating behaviours of the agents which can indirectly incorporate local avoidance but often this isn't enough to satisfy a collision free system, hence local avoidance can be 'layered' on top.\\

\section{Initial Research}
\begin{figure}
\includegraphics[scale=0.2]{../schedule/specialistSchedule.png}
\caption{Project Schedule}
\label{fig:schedule}
\end{figure}

First of all I made a time schedule in the form of a Gantt Chart, Figure\ref{fig:schedule} that I would follow in order to keep on track, as time management would be crucial in this project. For my initial research I looked into the various methods used for local avoidance and what techniques current systems used. The main methods I came across were Boids Flocking System\cite{CRBF}, Social Forces\cite{CLi2010CS}, PLE\cite{SG2010PLE},Continuum Crowds\cite{ATreuilleCS}, and Reciprocal Velocity Obstacle. Reciprocal Velocity Obstacle, RVO, seemed the most relevant and interesting method as it purely focused on object avoidance. RVO is a technique used in various systems including Goalem crowd plugin for Maya and Unreal Engine 4 which shows its relevance in industry.\\


\begin{figure}
\includegraphics[scale=0.13]{../umlDiagram/Agent_UML_diagram.png} 
\caption{UML class diagram}
\label{fig:umldiagram}
\end{figure}

Figure \ref{fig:umldiagram} shows my initial UML diagram of how I felt my code would be structured, to do this I looked into design patterns \cite{OODPWS}\cite{GPDSWS}.\\


\section{RVO}
RVO  is an agent based method for collision avoidance and was developed by Jur Van Den Burg et all \cite{JBerg2008RVO}. It is based upon a similar technique known as Velocity Obstacle (VO), and uses much of the same principle but with some critical tweaks, which was initially designed for motion planning of autonomous robots. However these techniques have been extended for use in crowd  simulation.

\subsection{How RVO Works}


\begin{figure}[b]
\includegraphics[scale=0.4]{images/rvoDiagram.png}
\caption{Velocity Obstacle cone}
\label{fig:rvoCone}
\end{figure}

\begin{center}
\begin{figure*}[t]
\centering
\includegraphics[scale=0.08]{images/RVO_circle.png}
\includegraphics[scale=0.08]{images/RVO_circle2.png}
\includegraphics[scale=0.08]{images/RVO_bounds_circ.png}
\includegraphics[scale=0.08]{images/RVO_bound_circ2.png}
\caption{RVO avoidance with 100 agents in a circle}
\label{fig:RVOcircle}
\end{figure*}
\end{center}


VO, which RVO is based from, works by the following:\\
\begin{center}
$VO^{A}_{B} (v_{B} ) = \{v_{A} | λ(p_{A} , v_{A} − v_{B} ) \cap B \oplus − A = \varnothing \}$
\end{center}
This translates to, the velocity obstacle $VO^{A}_{B} (v B )$ of agent \emph{B} to \emph{A} is the set of velocities $V_{A}$ for \emph{A} that will result in a collision with \emph{B} moving at velocity $V_{B}$. The idea is to find a new velocity for the agent \emph{A} that is outside the VO of \emph{B}. The difference with RVO and VO is instead of finding a velocity outside the VO of \emph{B} we find the average between the velocity outside and the current velocity \cite{JBerg2008RVO}\cite{AGuy2009CP}, this can be seen in figure \ref{fig:rvoCone}. This produces smoother movement of the agent and guarantees oscillation free movement which can be a common problem for local avoidance.\\


\subsection{RVO Implementation}
My implementation works by first creating the sample velocities according to the desired velocity. Then iterate through each velocity, within each iteration we also loop through all the neighbours. Inside this second loop we perform RVO. First we create the VO cone, then we check the current test velocity and if it intersects with the VO we find the time to collision, \emph{t}, and store it in an array for analysis later, however if the velocity does not collide we return a \emph{true} value that is stored in another array. If all agents tested against the current test velocities return \emph{true} we can exit all loops and accept the test velocity, however if this is not the case we move onto the next velocity and repeat. In the case of no velocity being directly acceptable we work out a penalty value for each velocity according to the value of \emph{t} to determine the best velocity\cite{DCherry2013RVO}\cite{JBerg2008CS}.\\

\section{Optimizations}
The first optimization I implemented was incorporating a hash table to speed up the neighbour search, rather than being of $O(n^{2})$ time complexity it became $O(n)$. A further optimization is to have the agents only search half the necessary neighbouring hash cells, when a neighbour is found the two agents add each other to their neighbour list. This means that half the number of neighbour queries have to be completed.\\
The next optimization was to refine RVO. During dense crowds the agents can become conservative with their movement, this is due to the shape of the VO cone restricting velocities that are temporarily acceptable. However the tip of the cone can be "cut off" thus shrinking the area of the VO\cite{AGuy2009CP}, this increases the number of available velocities for the agent. In the case of a velocity being selected due to its penalty value, we can multiply the velocity by \emph{t}, the time to collision, to allow the agent to change its speed. This is not perfect as it means some optimal velocities may be overshadowed however it is a quick method which is necessary for real-time applications.\\

\section{Conclusion and Future Work}
I am pleased with the outcome of the project as I feel I was successful in investigating local avoidance techniques focusing on real-time requirements. My implementation is proof off the success of my research as my simulation works with upto 200 agents smoothly. Future work will entail adding global navigation through the use of A* path finding as well as incorporating more complex behaviour such as a refined social forces model. I would also like to further optimize my simulation through GPGPU, certain parts of the program such as the nearest neighbour search on the hash table can be accomplished in parallel\cite{GGHT}. Even the implementation of RVO, with some tweaking, can be performed on the GPU, I started implementing this with the use of OpenGL Compute Shaders however could not finish this in the time frame.\\

\newpage
%\bibliographystyle{plainnat}
\bibliographystyle{plain_annote}
\bibliography{references}


Report word count: \emph{x}\\
Annotated bibliography word count: \emph{x}

\end{document}
